\documentclass[lettersize,journal]{IEEEtran}
\usepackage{amsmath,amsfonts}
\usepackage{algorithmic}
\usepackage{algorithm}
\usepackage{array}
\usepackage[caption=false,font=normalsize,labelfont=sf,textfont=sf]{subfig}
\usepackage{textcomp}
\usepackage{stfloats}
\usepackage{url}
\usepackage{verbatim}
\usepackage{graphicx}
\usepackage{cite}
\hyphenation{op-tical net-works semi-conduc-tor IEEE-Xplore}
% updated with editorial comments 8/9/2021

\begin{document}

\title{Aerial-supported Ground Search and Rescue Robot}

\author{Alfian Fadhlurrahman, Shailee Kampani, Edward Silvey, Alvin Zhafif Afilla}

% The paper headers
\markboth{University of Birmingham Intelligent Robotics}%
{Shell \MakeLowercase{\textit{et al.}}: A Sample Article Using IEEEtran.cls for IEEE Journals}

\maketitle

\begin{abstract}
This paper presents a collaborative approach to search and rescue, using both aerial search drones and ground rescue rovers.
\end{abstract}

\begin{IEEEkeywords}
Aerial Systems: Applications; Multi-Robot Systems; Search and Rescue Robots
\end{IEEEkeywords}

% I am not sure if the sections are correct and follow IEEE robotics format
\section{Introduction}
\IEEEPARstart{T}{his} section will outline and act as in intro to the file.  It seems to need the weird big letter as IEEE standards.  Could also be where we put link to repo.

\section{Drone Movement and Flight Control}
Section about how the drone works.

\section{Target Location Detection and Tagging}
Section about how victims are located and that location is sent to the ground section.

\section{Ground Robot Navigation, and Obstacle Avoidance}
Section about how ground robot is moved and how it avoids obstacles.

\section{Ground Robot Markov Localisation}
Section about how ground robot is moved and how it avoids obstacles.

\section{Conclusion}
The conclusion

\begin{thebibliography}{1}
\bibliographystyle{IEEEtran}


\bibitem{ref1}
Maybe this should be the repo.

\end{thebibliography}

\end{document}